\documentclass[a4paper,10pt]{article}
\usepackage[utf8]{inputenc} %pour utiliser tous les caracteres du clavier 
\usepackage[T1]{fontenc} %meme chose ici
\usepackage[francais]{babel} %pour ecrire en francais
\usepackage{listings} %pour citer du code

\usepackage{amsfonts} % pour utiliser les symboles de ensembles (reel...autre)
\usepackage{amsmath} %debut des package pour utiliser les formules de math
\usepackage{amssymb}
\usepackage{mathrsfs}
\usepackage[top=2cm, bottom=2cm, left=2.0cm, right=1.9cm]{geometry}

\usepackage{wrapfig}
\usepackage{graphicx} %pour charger des images
\usepackage[dvipsnames]{xcolor}
\usepackage{float}

\usepackage{textcomp}
\usepackage{url} % pour les url

 \usepackage{listings} % pour ajouter du code
 
\lstset{
language=C++,
basicstyle=\normalsize, % ou ça==> basicstyle=\scriptsize,
upquote=true,
aboveskip={1.5\baselineskip},
columns=fullflexible,
showstringspaces=false,
extendedchars=true,
breaklines=true,
showtabs=false,
showspaces=false,
showstringspaces=false,
identifierstyle=\ttfamily,
keywordstyle=\color[rgb]{0,0,1},
commentstyle=\color[rgb]{0.133,0.545,0.133},
stringstyle=\color[rgb]{0.627,0.126,0.941},
}

\title{NachOS : La pagination \\ Année 2013-2014}
\author{par Jerôme Barbier, Augustin Husson et David Levayer}
\date{\today}

\begin{document}
   \maketitle

  \begin{center}
    \includegraphics[width=10cm,height=10cm]{../partie3/robottrading.jpg}\\
    Rapport généré avec \LaTeX
  \end{center}
  \tableofcontents
  \newpage
  
  \textcolor{Violet}{\section{Adressage virtuelle par une table des pages}}
  \textcolor{Violet}{\section{Exécuter plusieurs programmes en même temps}}
  On met en place de manière classique l'appel système: 
  \begin{lstlisting}
  int ForkExec( char *s)
  \end{lstlisting}
  Pour rappel, on met en place un appel système en faisant les étapes suivantes :
  \begin{itemize}
   \item[1.] Dans \emph{syscall.h}, on écrit le prototype de notre appel système et on ajoute également une constante suivant le principe de nommage suivant:
   \begin{center}
   \emph{SC\_+nom\_de\_l'appel\_system}
   \end{center}
  Dans le cas présent on a rajouté ceci : 
  \begin{lstlisting}
   #define SC_ForkExec 20
   [...]
   int ForkExec( char *s);
  \end{lstlisting}
  \item[2.] On complète ensuite le fichier \emph{Start.s} qui implémente le fichier \emph{syscall.h} en assembleur. Ce qui donne ici :
  \begin{lstlisting}
   	.globl ForkExec
	.ent ForkExec

ForkExec: // nom de l'appel système
	addiu $2,$0,SC_ForkExec // ajout dans le registre 2 la valeur correspondant à l'appel sytème
	syscall // appel système
	j     $31
	.end ForkExec
  \end{lstlisting}
  \item[3.] Puis on s'efforce de complèter le fichier \emph{exception.cc} qui permet de gérer l'aiguillage vers les différents appels systèmes mise en place 
  tout au long du projet. Les différents cas étant gérer dans un switch, il suffit donc de rajouter notre ``cas'' comme ceci :
  \begin{lstlisting}
   case SC_ForkExec:{
     break;
   }
  \end{lstlisting}
  Bien entendu on tâchera de complèter ce cas là afin de répondre aux exigences du cahier des charges. On expliquera par la suite ce qu'il faut ajouter ici.
  
  \item[4.] Enfin il est nécessaire de créer un fichier de test qui permettra de tester cet appel sytème. On pensera à compiler à cet instant afin de bien 
  vérifier qu'aucun oublie dans ces étapes n'a été fait avant de complèter le code de \emph{exception.cc}
  \end{itemize}  
    \textcolor{NavyBlue}{\subsection{Un processus : un thread de plus haut niveau}}
    Comme l'indique le sujet de la sous-section, on va créer un processus en suivant le même procéder que lorsqu'on a créé des threads dans la partie III.
    On commence donc par créer deux fichiers : \emph{fork.cc} et \emph{fork.h}. Le second fichier ne présente guère de difficulter. Il contient simplement
    les signatures des fonctions qu'on a besoin. En l'occurance, nous avons besoin que d'une seule fonction publique :
    \begin{lstlisting}
     int do_UserFork(char * s);
    \end{lstlisting}
    Et maintenant voici le code que contient le fichier \emph{fork.cc}. On va bien sûr ajouter quelques explications en plus des commentaires du code qui 
    y sont déjà.
    
    \newpage
    \textcolor{TealBlue}{\subsubsection*{Le fichier fork.cc}}
    \begin{lstlisting}
#ifdef CHANGED
#include "fork.h"
#include "thread.h"
#include "addrspace.h"
#include "synch.h"
#include "system.h"
#include "console.h"

struct Serialisation{
	AddrSpace* space;
};

void StartProcess(int arg){
	Serialisation* restor = (Serialisation*) arg; // on restaure notre sérialisation 
	currentThread->space = restor->space; // on affecte le nouvel espace mémoir à notre nouveau processus
	currentThread->space->InitRegisters ();	// on réinitialise les registres
	currentThread->space->RestoreState ();	// on charge la table des pages des registres

	machine->Run ();		// on lance le processus
}

int do_UserFork(char *s){

	OpenFile *executable = fileSystem->Open (s);
	AddrSpace *space = new AddrSpace(executable); // création du nouvel espace mémoir du processus que l'on va mettre en place

	Thread* newThread = new Thread("newProcess"); // un processus est juste un thread avec un nouvel espace mémoir
  
	Serialisation* save = new Serialisation; // comme pour les threads, on sérialise l'espace mémoir qu'on souhaite affecter à notre processus
	save->space = space;

	newThread->Fork(StartProcess,(int)save); // on fork le processus père
	machine->SetNbProcess(machine->GetNbProcess()+1); // on incrémente de 1 le nbre de processus créé
	delete executable;
	currentThread->Yield(); // le processus père est mis en attente
	return 0;
}

#endif 

    \end{lstlisting}
    \newpage
  \textcolor{TealBlue}{\subsubsection*{Les explications}}
    Tout d'abord, il faut savoir que pour réaliser cette implémentation, il y avait deux solutions qui résident dans le fait d'utiliser ou non la méthode
    native \emph{Fork} de la classe \emph{Thread}. On rappel que la méthode Fork est destinée à créer un thread fils d'un processus père et non pas à 
    créer un autre processus père. 
    
    Ainsi, détourner l'objectif premier de cette méthode peut peut-être rebiffer les puristes. Afin de satisfaire les puritains et de montrer qu'il y avait plusieurs
    solutions possibles, deux implémentations différentes sont mises en place. 
    
    \begin{itemize}
     \item La première ne modifie pas la classe \emph{Thread} et utilise les méthodes
    déjà en place. Ce qui permet donc de ne pas rajouter du code supplémentaire dans les classes du système. Cependant pour contourner les restrictions
    de la méthode \emph{Fork}, une sérialisation a été mise en place. Elle permet de sauvegarder le nouvel espace mémoir qu'on souhaite alloué au 
    processus créé.
    
    \item La seconde méthode consiste à faire une surcharge de la méthode \emph{Fork} en ajoutant comme paramêtre l'espace mémoir que le nouveau processus
    doit occuper. Ensuite cette nouvelle méthode fait exactement la même chose que son original au détail près relatif à la mémoire allouée.
    \end{itemize}
  Dans le code ci-dessus, c'est la première méthode qui a été mise en place pour faire fonctionner l'appel système. Notez bien que ce choix est totalement 
  arbitraire. 
  Par ailleurs, si on souhaitait mettre en place la deuxième méthode, il suffirait de remplacer la ligne : 
  \begin{lstlisting}
   newThread->Fork(StartProcess,(int)save);
  \end{lstlisting}
  par : 
  \begin{lstlisting}
   newThread->ForkExec(StartProcess,NULL,(int)space);
  \end{lstlisting}
  Et bien entendu dans la procédure \emph{StartProcess}, il faudra commenter les deux premières lignes qui ne sont plus utiles pour la $2^{eme}$ solution.
  
  On remarquera que le $3^{eme}$ argument de la méthode \emph{ForkExec} est de type \emph{int} et pas de type \emph{AddrSpace*}. Cette spécificité est une 
  astuce pour éviter des appels réccursifs à la bibliothèque \emph{addrspace.h} dans la classe \emph{Thread}.
  
  \textcolor{TealBlue}{\subsubsection*{La completion du fichier execption.cc}}
  Maintenant que tout est en place pour faire enfin notre appel système, il est grand temps de faire le pont entre l'utilisateur et le système et donc
  de complêter le fichier \emph{exception.cc}. Sans plus attendre voici ce qui est rajouté : 
  \begin{lstlisting}
          case SC_ForkExec:{
            int arg = machine->ReadRegister(4);
            char* to = new char[MAX_STRING_SIZE+1]; // buffer le +1 permet d'ajouter le caractere de fin de chaine
            synchconsole->CopyStringFromMachine(arg, to, MAX_STRING_SIZE);
            int res = do_UserFork(to);

            ASSERT(res==0);
            break;
          }   
  \end{lstlisting}
  Donc classiquement on retrouve les arguments de l'appel système dans le registre 4. Et comme toujours, on a juste l'adresse de l'argument qui est donc
  de type int. Hors, on sait que l'argument est une chaîne de caractère. Il faut donc refaire la même procédure que pour l'appel 
  système \emph{SynchPutString}. Et c'est donc ce qui est fait là avec la variable\emph{char* to} et l'appel à la méthode \emph{CopyStringFromMachine}.
  
  \newpage
  \textcolor{NavyBlue}{\subsection{Main!! Ne vas pas trop vite}}
  Après avoir testé ce nouvel appel système, on se rend rapidement compte que le nouveau processus a à peine le temps de s'executer que la machine nachOS
  est arrêté. Il faut donc pouvoir empêcher que l'appel à la méthode \emph{Halt} qui met stoppe nachOS de s'exécuter tant que tous les processus n'ont pas
  fini leur exécution.
  
  Dans ce but, on met en place un compteur de processus qui se fait donc dans la classe \emph{machine}. Ce compteur est l'attribut \emph{nbProcess}. Cet
  attribut étant privé, un getter et setter (\emph{GetNbProcess} et \emph{SetNbProcess}) sont mis en place pour le modifier.
  
  De cette façon à chaque création d'un processus (i.e à chaque appel système \emph{ForkExec}) on incrémente ce compteur de 1. Et lorsqu'un processus est 
  sur le point de s'arrêter : 
  \begin{itemize}
   \item On teste tout d'abord combien il y a de processus en cours d'exécution. Le compteur étant initialement mis à 0, si le compteur est supérieur 
   strict à 0, alors y a plus de deux processus qui existent.
   \item Dans le cas où il y a plus de deux processus, celui qui souhaite s'arrêter, décrémente alors de 1 le compteur et exécute la méthode \emph{Finish()}
   \item Dans le cas où il y a un seul processus, on fait alors un appel à la méthode \emph{Halt}.
  \end{itemize}
  Bien entendu ce code se met après celui qui vérifie que tous les threads d'un processus se sont arrêtés. Et pour conclure ces explications
  voici le code qui y correspond :
  \begin{lstlisting}
          case SC_Halt:{
            DEBUG('a', "Shutdown, initiated by user program.\n");
            while(currentThread->space->NbreThread()>1) // tant qu'il y a plus que un thread on reste bloquer
              currentThread->space->LockEndMain();
            
            if(machine->GetNbProcess() > 0){
              machine->SetNbProcess(machine->GetNbProcess()-1);
              currentThread->Finish();
            }
            interrupt->Halt();
            break;
          }   
  \end{lstlisting}

\end{document}
